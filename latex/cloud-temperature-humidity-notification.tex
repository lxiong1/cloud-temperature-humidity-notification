\documentclass{article}
\usepackage[utf8]{inputenc}
\usepackage{listings}
\usepackage{graphicx}
\usepackage{float}

\setlength{\parindent}{0pt}
\lstset{
    basicstyle=\ttfamily\footnotesize,
    frame=single,
    xleftmargin=4pt,
    xrightmargin=4pt,
    breaklines=true
}

\title{Cloud Temperature \& Humidity Notification System}
\author{Lue Xiong}

\begin{document}

\maketitle
\newpage

\tableofcontents
\newpage

\obeylines

\section{Context}
The Cloud Temperature \& Humidity Notification System is about an IoT system that gives the ability to notify user(s) of temperature and humidity fluctuations within their home through the usage of a Simple Message Service, which is also known as SMS. Though this is the main concern of the system, it also allows users to visualize their daily climate averages in an interactive graph. The graph is automatically updated for the users to view whenever they want to in the web. Behind the scenes, most computation are abstracted away from the user by using Google Cloud Platform. The initiation of these functionalities start from the temperature and humidity readings being published by a Particle Argon.\\

Working within the limitations of a small apartment, money, and of time alloted for the project, I am unable to realize the full potential of the system. This project represents a single IoT device that enacts the above mentioned functionalities. One can imagine however, being able to send in-home area location data and climate readings with a multiple of these IoT devices scattered across a home with multiple rooms and stories. A user would be notified where in the house and when the climate has reached unwanted levels based on how they have configured the device. They would then be able to see data points for each specified area of the home. That is the aspiration. However this particular projects seeks an minimum viable product, which is notifying users via SMS notification and graphing average climate per day for viewing averages over days.

\section{Problem \& Market Research}
From market researching, there are a couple of products that are similar:
\begin{itemize}
    \item ThermoPro TP50
    \item Govee Temperature Humidity Monitor
    \item Proteus AMBIO
\end{itemize}

Differentiation: Price and Device Status Alerts

The Cloud Temperature \& Humidity Notification System seeks to solve the issue of the middle ground for home automation.

\section{Solution Architecture \& Design Approach}
\subsection{SMS Notification Rules \& Trigger Criteria}
\section{Process}
\subsection{Hardware Setup}
\subsection{Implementation With Particle Argon \& Ecosystem}
\subsection{Implementation With Google Cloud Platform }
\subsubsection{Purpose of Tools Within Platform}
\subsubsection{SMS Notification}
\subsubsection{Climate Data Graph}

% \begin{minipage}[c]{\textwidth}
% \begin{lstlisting}
% function createSpreadsheetOnChangeTrigger() {
%   let spreadsheet = SpreadsheetApp.getActive()

%   ScriptApp.newTrigger('appendAverageCalculation')
%     .forSpreadsheet(spreadsheet)
%     .onChange()
%     .create();
% }
% \end{lstlisting}
% \end{minipage}\ \\

% \begin{figure}[H]
% \center
% \includegraphics[width=\textwidth]{images/particle-argon-gcp-sheets-script.png}
% \caption{Google Sheets Script}
% \label{fig:script}
% \end{figure}

\section{Discussion of Results}

\section{Conclusion}

\end{document}
